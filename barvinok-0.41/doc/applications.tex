\section{\texorpdfstring{Applications included
in the \protect\ai[\tt]{barvinok} distribution}
{Applications included in the barvinok distribution}}
\label{a:usage}

\index{barvinok@{\tt  barvinok}!availability}
{\sloppy 
This section describes some application programs
provided by the \barvinok/ library,
available from \url{http://barvinok.gforge.inria.fr/}.
For compilation instructions we refer to the \verb+README+ file
included in the distribution.
}

Common option to all programs:\\
\begin{tabular}{lll}
\ai[\tt]{--version} & \ai[\tt]{-V} & print version
\\
\ai[\tt]{--help} & \ai[\tt]{-?} & list available options
\end{tabular}

\subsection{\texorpdfstring{\protect\ai[\tt]{barvinok\_count}}{barvinok\_count}}

The program \ai[\tt]{barvinok\_count} enumerates a
non-parametric polytope.  It takes one polytope
in \PolyLib/ notation as input and prints the number
of integer points in the polytope.
The \PolyLib/ notation corresponds to the internal
representation of \ai[\tt]{Polyhedron}s as explained
in Section~\ref{a:existing}.
\index{input format!constraints}
The first line of the input contains the number of rows
and the number of columns in the \ai[\tt]{Constraint} matrix.
The rest of the input is composed of the elements of the matrix.
Recall that the number of columns is two more than the number
of variables, where the extra first columns is one or zero
depending on whether the constraint is an inequality ($\ge 0$)
or an equality ($= 0$).  The next columns contain
the coefficients of the variables and the final column contains
the constant in the constraint.
E.g., the set 
$S = \lb\, s \mid s \geq 0 \wedge  2 s \leq 13 \, \rb$
from \citeN[Example~38 on page~134]{Verdoolaege2005PhD}
corresponds to the following input and
output.
\begin{verbatim}
> cat S
2 3

1 1 0
1 -2 13
> ./barvinok_count  < S
POLYHEDRON Dimension:1
           Constraints:2  Equations:0  Rays:2  Lines:0
Constraints 2 3
Inequality: [   1   0 ]
Inequality: [  -2  13 ]
Rays 2 3
Vertex: [   0 ]/1
Vertex: [  13 ]/2
   7 
\end{verbatim}
\index{PolyLib@{\tt  PolyLib}!version 5.22.0 or newer}
Note that if you use \PolyLib/ version 5.22.0 or newer then the output
may look slightly different as the computation of the \ai[\tt]{Rays}
may have been postponed to a later stage.
The program \ai[\tt]{latte2polylib.pl} can be used to
convert a polytope from \ai[\tt]{LattE} \shortcite{latte1.1}
notation to \PolyLib/ notation.

\index{input format!vertices}
As an alternative to the constraints based input, the input polytope
may also be specified by its \ai[\tt]{Ray} matrix.
The first line of the input contains the single word \ai[\tt]{vertices}.
The second line contains the number of rows
and the number of columns in the \ai[\tt]{Ray} matrix.
The rest of the input is composed of the elements of the matrix.
Recall that the number of columns is two more than the number
of variables, where the extra first columns is one or zero
depending on whether the ray is a line or not.
The next columns contain
the numerators of the coordinates and the final column contains
the denominator of the vertex or $0$ for a ray.
E.g., the above set can also be described as
\begin{verbatim}
vertices

2 3

1 0 1
1 13 2
\end{verbatim}

\subsection{\texorpdfstring{\protect\ai[\tt]{barvinok\_enumerate}}
{barvinok\_enumerate}}

The program \ai[\tt]{barvinok\_enumerate} enumerates a
parametric polytope as a \psp/ or \rgf/.  It takes two polytopes in \PolyLib/
notation as input, optionally followed by a list of parameter
names.
The two polytopes refer to arguments \verb+P+ and \verb+C+
of the corresponding function. (See Section~\ref{a:counting:functions}.)
The following example was taken by \shortciteN{Loechner1999}
from \shortciteN[Chapter II.2]{Loechner97phd}.
\begin{verbatim}
> cat loechner 
# Dimension of the matrix: 
7 7 
# Constraints: 
# i j k P Q cte 
1 1 0 0 0 0 0 # 0 <= i 
1 -1 0 0 1 0 0 # i <= P 
1 0 1 0 0 0 0 # 0 <= j 
1 1 -1 0 0 0 0 # j <= i
1 0 0 1 0 0 0 # 0 <= k 
1 1 -1 -1 0 0 0 # k <= i-j 
0 1 1 1 0 -1 0 # Q = i + j + k

# 2 parameters, no constraints. 
0 4
> ./barvinok_enumerate < loechner 
POLYHEDRON Dimension:5
           Constraints:6  Equations:1  Rays:5  Lines:0
Constraints 6 7
Equality:   [   1   1   1   0  -1   0 ]
Inequality: [   0   1   1   1  -1   0 ]
Inequality: [   0   1   0   0   0   0 ]
Inequality: [   0   0   1   0   0   0 ]
Inequality: [   0  -2  -2   0   1   0 ]
Inequality: [   0   0   0   0   0   1 ]
Rays 5 7
Ray:    [   1   0   1   1   2 ]
Ray:    [   1   1   0   1   2 ]
Vertex: [   0   0   0   0   0 ]/1
Ray:    [   0   0   0   1   0 ]
Ray:    [   1   0   0   1   1 ]
POLYHEDRON Dimension:2
           Constraints:1  Equations:0  Rays:3  Lines:2
Constraints 1 4
Inequality: [   0   0   1 ]
Rays 3 4
Line:   [   1   0 ]
Line:   [   0   1 ]
Vertex: [   0   0 ]/1
         - P + Q  >= 0
         2P - Q  >= 0
          1 >= 0

( -1/2 * P^2 + ( 1 * Q + 1/2 )
 * P + ( -3/8 * Q^2 + ( -1/2 * {( 1/2 * Q + 0 )
} + 1/4 )
 * Q + ( -5/4 * {( 1/2 * Q + 0 )
} + 1 )
 )
 )
         Q  >= 0
         P - Q  -1 >= 0
          1 >= 0

( 1/8 * Q^2 + ( -1/2 * {( 1/2 * Q + 0 )
} + 3/4 )
 * Q + ( -5/4 * {( 1/2 * Q + 0 )
} + 1 )
 )
\end{verbatim}
The output corresponds to 
$$
\begin{cases}
\makebox[0pt][l]{$-\frac 1 2 P^2 + P Q + \frac 1 2 P - \frac 3 8 Q^2
+ \left( \frac 1 4 - \frac 1 2 \left\{ \frac 1 2 Q \right\} \right)
	       	Q + 1 
- \frac 5 4 \left\{ \frac 1 2 Q \right\}$} \\
&
\hbox{if $P \le Q \le 2 P$}
\\
\frac 1 8 Q^2 + 
\left( \frac 3 4 - \frac 1 2 \left\{ \frac 1 2 Q \right\} \right)
- \frac 5 4 \left\{ \frac 1 2 Q \right\}
\qquad
\qquad
\qquad
\qquad
\qquad
&
\hbox{if $0 \le Q \le P-1$}
.
\end{cases}
$$
The following is an example of Petr Lison\u{e}k\index{Lison\u{e}k, P.}.
\begin{verbatim}
> cat petr
4 6
1 -1 -1 -1 1 0
1 1 -1 0 0 0
1 0 1 -1 0 0
1 0 0 1 0 -1

0 3
n
> ./barvinok_enumerate --series < petr
POLYHEDRON Dimension:4
           Constraints:5  Equations:0  Rays:5  Lines:0
Constraints 5 6
Inequality: [  -1  -1  -1   1   0 ]
Inequality: [   1  -1   0   0   0 ]
Inequality: [   0   1  -1   0   0 ]
Inequality: [   0   0   1   0  -1 ]
Inequality: [   0   0   0   0   1 ]
Rays 5 6
Ray:    [   1   1   1   3 ]
Ray:    [   1   1   0   2 ]
Ray:    [   1   0   0   1 ]
Ray:    [   0   0   0   1 ]
Vertex: [   1   1   1   3 ]/1
POLYHEDRON Dimension:1
           Constraints:1  Equations:0  Rays:2  Lines:1
Constraints 1 3
Inequality: [   0   1 ]
Rays 2 3
Line:   [   1 ]
Vertex: [   0 ]/1
(n^3)/((1-n) * (1-n) * (1-n^2) * (1-n^3))
\end{verbatim}

Options:\\
\begin{tabular}{llp{0.7\textwidth}}
\ai[\tt]{--floor} & \ai[\tt]{-f} & 
convert \ai[\tt]{fractional}s to \ai[\tt]{flooring}s
\\
\ai[\tt]{--convert} & \ai[\tt]{-c} & 
convert \ai[\tt]{fractional}s to \ai[\tt]{periodic}s
\\
\ai[\tt]{--series} & \ai[\tt]{-s} & 
compute \rgf/ instead of \psp/
\\
\ai[\tt]{--explicit} & \ai[\tt]{-e} & 
convert computed \rgf/ to a \psp/
\end{tabular}

\subsection{\texorpdfstring{\protect\ai[\tt]{barvinok\_enumerate\_e}}
{barvinok\_enumerate\_e}}

The program \ai[\tt]{barvinok\_enumerate\_e} enumerates a
parametric projected set.  It takes a single polytope in \PolyLib/
notation as input, followed by two lines indicating the number
or existential variables and the number of parameters and
optionally followed by a list of parameter names.
The syntax for the line indicating the number of existential
variables is the letter \verb+E+ followed by a space and the actual number.
For indicating the number of parameters, the letter \verb+P+ is used.
The following example corresponds to 
\citeN[Example~36 on page~129]{Verdoolaege2005PhD}.
\begin{verbatim}
> cat projected 
5 6
#   k   i   j   p   cst
1   0   1   0   0   -1
1   0   -1  0   0   8
1   0   0   1   0   -1
1   0   0   -1  1   0
0   -1  6   9   0   -7

E 2
P 1
> ./barvinok_enumerate_e <projected 
POLYHEDRON Dimension:4
           Constraints:5  Equations:1  Rays:4  Lines:0
Constraints 5 6
Equality:   [   1  -6  -9   0   7 ]
Inequality: [   0   1   0   0  -1 ]
Inequality: [   0  -1   0   0   8 ]
Inequality: [   0   0   1   0  -1 ]
Inequality: [   0   0  -1   1   0 ]
Rays 4 6
Vertex: [  50   8   1   1 ]/1
Ray:    [   0   0   0   1 ]
Ray:    [   9   0   1   1 ]
Vertex: [   8   1   1   1 ]/1
exist: 2, nparam: 1
         P  -3 >= 0
          1 >= 0

( 3 * P + 10 )
         P  -1 >= 0
         - P + 2 >= 0

( 8 * P + 0 )
\end{verbatim}

Options:\\
\begin{tabular}{llp{0.7\textwidth}}
\ai[\tt]{--floor} & \ai[\tt]{-f} & 
convert \ai[\tt]{fractional}s to \ai[\tt]{flooring}s
\\
\ai[\tt]{--convert} & \ai[\tt]{-c} & 
convert \ai[\tt]{fractional}s to \ai[\tt]{periodic}s
\\
\ai[\tt]{--omega} & \ai[\tt]{-o} & 
use \Omegalib/ as a preprocessor
\\
\ai[\tt]{--pip} & \ai[\tt]{-p} & 
\begin{minipage}[t]{0.7\textwidth}
\raggedright
call \ai[\tt]{barvinok\_enumerate\_pip} instead of \ai[\tt]{barvinok\_enumerate\_e}
\end{minipage}
\\
\ai[\tt]{--isl} & \ai[\tt]{-i} & 
\raggedright
call \ai[\tt]{barvinok\_enumerate\_isl} instead of \ai[\tt]{barvinok\_enumerate\_e}
\end{tabular}

\subsection{\texorpdfstring{\protect\ai[\tt]{barvinok\_union}}
{barvinok\_union}}

The program \ai[\tt]{barvinok\_union} enumerates a \ai{union} of
parametric polytopes.  It takes as input the number of parametric
polytopes in the union, the polytopes in combined data and
parameter space in \PolyLib/ notation, the context in parameter space
in \PolyLib/ notation and optionally a list of parameter names.

Options:\\
\begin{tabular}{llp{0.7\textwidth}}
\ai[\tt]{--series} & \ai[\tt]{-s} & 
compute \rgf/ instead of \psp/
\end{tabular}

\subsection{\texorpdfstring{\protect\ai[\tt]{barvinok\_ehrhart}}
{barvinok\_ehrhart}}

\sindex{Ehrhart}{quasi-polynomial}
The program \ai[\tt]{barvinok\_ehrhart} computes the
\ai{Ehrhart quasi-polynomial} of a polytope $P$, i.e., a quasi-polynomial
in $n$ that evaluates to the number of integer points in the dilation
of $P$ by a factor $n$.
The input is the same as that of \ai[\tt]{barvinok\_count}, except that
it may be followed by the variable name.
The functionality is the same as running \ai[\tt]{barvinok\_enumerate}
on the cone over $P$ placed at $n=1$.

Options:\\
\begin{tabular}{llp{0.7\textwidth}}
\ai[\tt]{--floor} & \ai[\tt]{-f} & 
convert \ai[\tt]{fractional}s to \ai[\tt]{flooring}s
\\
\ai[\tt]{--convert} & \ai[\tt]{-c} & 
convert \ai[\tt]{fractional}s to \ai[\tt]{periodic}s
\\
\ai[\tt]{--series} & \ai[\tt]{-s} & 
compute \ai{Ehrhart series} instead of \ai{Ehrhart quasi-polynomial}
\end{tabular}

\subsection{\texorpdfstring{\protect\ai[\tt]{polyhedron\_sample}}
{polyhedron\_sample}}

The program \ai[\tt]{polyhedron\_sample} takes a polytope
in \PolyLib/ notation and prints an integer point in the polytope
if there is one.  The point is computed using
\ai[\tt]{Polyhedron\_Sample}.

\subsection{\texorpdfstring{\protect\ai[\tt]{polytope\_scan}}
{polytope\_scan}}

The program \ai[\tt]{polytope\_scan} takes a polytope in
\PolyLib/ notation and prints a list of all integer points in the polytope.
Unless the \ai[\tt]{--direct} options is given, the order is based
on the \ai{reduced basis} computed with
\ai[\tt]{Polyhedron\_Reduced\_Basis}.

Options:\\
\begin{tabular}{llp{0.7\textwidth}}
\ai[\tt]{--direct} & \ai[\tt]{-d} & 
list the points in the lexicographical order
\end{tabular}

\subsection{\texorpdfstring{\protect\ai[\tt]{lexmin}}{lexmin}}

The program \ai[\tt]{lexmin} implements an algorithm for performing
\indac{PIP} based on \rgf/s and provides an alternative for the
technique of \shortciteN{Feautrier88parametric}, which is based
on \ai{cutting plane}s \shortcite{Gomory1963}.
The input is the same as that of the \ai[\tt]{example} program
from \piplib/ \cite{Feautrier:PIP}, except that the value
for the ``\ai{big parameter}'' needs to be $-1$, since there is
no need for big parameters, and it does not read any options
from the input file.

\subsection{\texorpdfstring{\protect\ai[\tt]{barvinok\_summate}}
{barvinok\_summate}}

Given a \psp/ in \isl/ format,
the program \ai[\tt]{barvinok\_summate} computes the \ai{sum} of
the piecewise quasi-polynomial evaluated in all (integer) values of
the variables.  The result is an expression in the parameters.
Note that \ai[\tt]{barvinok\_enumerate} and \ai[\tt]{barvinok\_enumerate\_e}
can produce \psp/s when given the \ai[\tt]{-I} option, but they will
have only parameters and no variables.

For example
\begin{verbatim}
> cat square_p3.pwqp
[n] -> { [x, y] -> x * y :
         n >= -9 + 3x and x >= 2 and y >= 4 and y <= 5 }
> ./barvinok_summate < square_p3.pwqp
[n] -> { (45 + 63/2 * [(n)/3] + 9/2 * [(n)/3]^2) : n >= -3 }
\end{verbatim}

Options:\\
\begin{tabular}{llp{0.7\textwidth}}
\ai[\tt]{--summation} & & 
specifies which summation method to use;
\ai[\tt]{box} refers to the method of
\shortciteN[Section~4.5.4]{Verdoolaege2005PhD},
\ai[\tt]{bernoulli} refers to the method of
\autoref{s:nested:exact},
\ai[\tt]{euler} refers to the method of
\autoref{s:euler},
and \ai[\tt]{laurent} refers to the method of
\autoref{s:laurent}.
\end{tabular}

\subsection{\texorpdfstring{\protect\ai[\tt]{barvinok\_bound}}
{barvinok\_bound}}

Given a \psp/ in \isl/ format,
the program \ai[\tt]{barvinok\_bound} computes an \ai{upper bound}
(or \ai{lower bound}) for
the values attained by the piecewise quasi-polynomial
over all (integer) values of the variables.
The result is an expression in the parameters.
Note that \ai[\tt]{barvinok\_enumerate} and \ai[\tt]{barvinok\_enumerate\_e}
can produce \psp/s when given the \ai[\tt]{-I} option, but they will
have only parameters and no variables.

\begin{verbatim}
> cat devos.pwqp
[U] -> { [V] -> ((1/3 * U + 2/3 * V) - [(U + 2V)/3]) :
         2V >= -3 - U and 2V <= -U and U >= 0 and U <= 10 }
> ./barvinok_bound < devos.pwqp
[U] -> { max(2/3) : U <= 10 and U >= 0 }
\end{verbatim}

Options:\\
\begin{tabular}{llp{0.7\textwidth}}
\ai[\tt]{--lower} & &
compute lower bound instead of upper bound
\end{tabular}

\subsection{\texorpdfstring{\protect\ai[\tt]{polytope\_minimize}}
{polytope\_minimize}}

The program \ai[\tt]{polytope\_minimize} has been superseded
by \isl/'s \ai[\tt]{isl\_polyhedron\_minimize}.

\subsection{\texorpdfstring{\protect\ai[\tt]{polyhedron\_integer\_hull}}
{polyhedron\_integer\_hull}}

The program \ai[\tt]{polyhedron\_integer\_hull} takes a polyhedron
in \PolyLib/ notation and
prints its \ai{integer hull}.
The integer hull is computed as explained in \autoref{s:integer:hull}.

\subsection{\texorpdfstring{\protect\ai[\tt]{polytope\_lattice\_width}}
{polytope\_lattice\_width}}

The program \ai[\tt]{polytope\_lattice\_width} computes
the \ai{lattice width} of a parametric polytope.
The input is the same as that of \ai[\tt]{barvinok\_enumerate}.
The lattice width is computed as explained
in \autoref{s:width}.

Options:\\
\begin{tabular}{llp{0.7\textwidth}}
\ai[\tt]{--direction} & \ai[\tt]{-d} & 
print the lattice width directions
\end{tabular}
